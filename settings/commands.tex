% ====================== PRAKTISCHE BEFEHLE ======================
% (Entfernen der %-Zeichen zum Aktivieren)

%%% Materialien und Normen
\newcommand{\stahlguete}[1]{\textbf{1.${#1}$}}          % Stahlgüteformatierung, z.B. \stahlgüte{0578}
\newcommand{\din}[1]{DIN {#1}}                         % Normenreferenz, z.B. \din{EN 10025-2}

%%% Einheiten (nutzt siunitx)
\newcommand{\MPa}[1]{\SI{#1}{\mega\pascal}}            % Druckeinheit, z.B. \MPa{350}
\newcommand{\kNm}[1]{\SI{#1}{\kilo\newton\meter}}      % Drehmoment, z.B. \kNm{12.5}

%%% Werkstoffkunde
\newcommand{\alusil}{\text{AlSi}7\text{Mg0.3}}         % Aluminium-Silizium-Legierung
\newcommand{\cfk}[1]{CFK-{#1}}                         % Kohlefaserformat, z.B. \cfk{T300}

%%% Technische Diagramme (TikZ)
\newcommand{\koordsystem}{%                            % Standard-Koordinatensystem
  \begin{tikzpicture}[scale=0.6]
    \draw[thick,->] (0,0) -- (2,0) node[right]{$x$};
    \draw[thick,->] (0,0) -- (0,2) node[above]{$y$};
    \draw[thick,->] (0,0) -- (-1.2,-1.2) node[below left]{$z$};
  \end{tikzpicture}}

%%% Mechanische Elemente
\newcommand{\lager}[2]{%                               % Standard-Lagerdarstellung
  \begin{tikzpicture}
    \node[draw,circle,minimum size=8mm] at (#1,#2) {};
    \draw[thick] (#1-0.4,#2) -- (#1+0.4,#2);
  \end{tikzpicture}}


%%% Finite-Elemente-Methode
%\newcommand{\fea}{\textsc{Fem}\xspace}                  % FEM-Abkürzung mit automat. Leerzeichen
%\newcommand{\meshparams}[3]{%                          % Netzparameter
%  \begin{itemize}[noitemsep]
%    \item Elementgröße: #1 mm
%    \item Knotenanzahl: #2
%    \item Verfeinerung: #3
%  \end{itemize}
%}