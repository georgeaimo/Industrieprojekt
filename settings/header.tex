% ====================== GRUNDLAGEN ======================
% Sprache und Schrift
\usepackage[T1]{fontenc}          % Bessere Unterstützung für europäische Zeichensätze
\RedeclareSectionCommand[
  font=\Huge\normalfont\rmfamily\bfseries, % Adjust size (options: \Huge, \LARGE, \Large, etc.)
  beforeskip=2cm, % Space before the chapter title
  afterskip=1cm % Space after the chapter title
]{chapter}


\usepackage[utf8]{inputenc}       % UTF-8 Zeichenkodierung
\usepackage[ngerman]{babel}       % Deutsche Sprachspezifika (Silbentrennung, Übersetzungen)
\usepackage{grffile}              % Erlaubt Sonderzeichen in Bilddateipfaden
\renewcommand{\baselinestretch}{1} % Zeilenabstand (1 = Standard)


\usepackage{scrhack}

% ====================== MATHEMATIK ======================
\usepackage{amsmath}              % Verbesserte Mathematikumgebungen (equation, align)
\usepackage{amsfonts}             % Zusätzliche Mathe-Schriftarten (mathbb)
\usepackage{amssymb}              % Mathematische Symbole
\usepackage[makeroom]{cancel}     % Durchstreichen von Termen (\cancel{...})
\usepackage{lscape}               % Querformat-Seiten für umfangreiche Formeln/Tabellen

% ====================== GRAFIK & FARBEN =================
\usepackage[usenames,dvipsnames,svgnames,table]{xcolor} % Erweiterte Farben
\usepackage{graphicx}             % Standardgrafikeinbindung
%\usepackage[demo]{graphicx}      % Dummies als Bilder +++ besser beim compilieren +++
\usepackage{subcaption}           % Unterbilder mit eigenen Beschriftungen
\usepackage[font=sf]{caption}     % Serifenlose Schrift in Bildunterschriften
\usepackage{rotating}             % Drehung von Grafiken/Text (sideways-Umgebung)
\usepackage{float}                % Präzise Bildplatzierung (H-Option)
\usepackage{tikz}                 % Vektorgrafiken zeichnen
\usetikzlibrary{                  % TikZ-Erweiterungen
    shapes.geometric, 
    arrows.meta,
    positioning
    } 

% ====================== TABELLEN & LISTEN ===============
\usepackage{booktabs}             % Professionelle Tabellenformatierung
\usepackage{tabularx}             % Tabellen mit automatischer Spaltenanpassung
\usepackage{makecell}             % Mehrzeilige Tabellenzellen
\usepackage{multirow}             % Verbundene Zeilen in Tabellen
\usepackage{enumitem}             % Anpassbare Aufzählungen (Listen)

% ====================== SEITENLAYOUT ====================
\usepackage[headsepline=0pt]{scrlayer-scrpage} % KOMA-Script-kompatibles Paket für Kopf- und Fußzeilen
\clearpairofpagestyles        % Löscht alle Standard-Stile
\cfoot{\pagemark}             % Seitenzahl in der Fußzeile zentriert
\KOMAoptions{parskip=full}   % Absatzabstände statt Einrückung
\setparsizes{10pt plus1pt}{0pt}{1em plus1fil} % Äquivalent zu deiner `parskip`-Konfiguration

% ====================== HYPERLINKS & VERWEISE ===========
\usepackage[pdftex,pdfauthor={Lars Meinke}]{hyperref} % Interaktive Links im PDF
\usepackage[ngerman]{cleveref}    % Intelligente Verweise (\cref)
\usepackage{bookmark}             % Verbesserte PDF-Lesezeichen

% ============== Nummerierung ==============
\usepackage{chngcntr}
\counterwithin{figure}{chapter}  % Figures numbered as 1.1, 1.2, 2.1, etc.
\counterwithin{table}{chapter}   % Tables numbered similarly
\counterwithin{equation}{chapter}% Equations numbered per chapter

% ====================== LITERATUR & ZITATE ==============
\usepackage[numbers]{natbib}      % Numerisches Zitieren (IEEE-Stil)
%\usepackage{acronym}  
\usepackage{acro}            % Abkürzungsverzeichnis

% ====================== SONSTIGE PAKETE =================
\usepackage{siunitx}              % SI-Einheiten (z.B. \SI{10}{\meter})
\usepackage{verbatim}             % Erweiterte Code-Umgebungen
\usepackage{pdfpages}             % PDF-Seiten direkt einbinden
\usepackage[title]{appendix}      % Anhangformatierung

% ====================== DESIGNANPASSUNGEN ===============
\usepackage{xcolor}                % Für Farben (falls nicht schon geladen)
\definecolor{FHBlau}{RGB}{16,46,92} % Corporate Color

% Customize chapter/section title colors with KOMA-Script
\addtokomafont{chapter}{\color{FHBlau}}      % Kapitelüberschriften
\addtokomafont{section}{\color{FHBlau}}      % Abschnittsüberschriften
\addtokomafont{subsection}{\color{FHBlau}}   % Unterabschnitte

% ====================== HYPHENATION =====================
\hyphenation{Über-trag-ungs-funk-tion Differ-en-tial-glei-chung}
\hyphenpenalty=20                 % Silbentrennungseinstellungen
\emergencystretch=1em

